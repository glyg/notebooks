\documentclass[12pt]{article}

\usepackage[french]{babel}
\usepackage{graphicx}
\usepackage{amsmath}
\usepackage{amsfonts}
\usepackage{amssymb}
\usepackage[small]{subfigure}
\usepackage{xspace}
\usepackage{upgreek}
\usepackage{wrapfig}


\graphicspath{{illus/}}

\newcommand{\TMax}{\theta_{max}}
\newcommand{\Tmoy}{<\theta>}

\begin{document}

\title{Simulation de la croissance et de l'élongation du fuseau}

\author{Guillaume Gay}
\date{mars 2007}


\section{Problème}

On étudie les variations d'angles du fuseau dans la cellule au cours
de la métaphase.  Que l'on mesure l'angle maximum ($\theta_{max}$) ou
l'angle moyen ($\Tmoy$) au cours de la mitose, on observe une ``zone
d'exclusion'' dans le plan (angle, durée de la métaphase). On
interprète cette zone d'exclusion comme un délai dû au SOC.

Cependant, on pourrait penser que cette observation est simplement due
à un biais statistique:
Dans l'hypothèse où les variations d'angle en métaphse sont
stochastiques, l'augmentation de la durée $t_{II}$  en métatphase augmente
la probabilité d'observer un angle maximum important. En revanche,
l'angle moyen devrait tendre vers 0 avec l'augmentation de $t_{II}$.


\section{Hypothèses}

\section{Pas de membrane nucléaire}

\section{Membrane nucléaire rigide}




\end{document}
